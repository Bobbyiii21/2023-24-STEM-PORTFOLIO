\documentclass[12pt]{report}
\usepackage{times}
\usepackage{graphicx}
%set font of document to helvetica
\renewcommand{\familydefault}{\rmdefault}

\usepackage[utf8]{inputenc}
\usepackage[english]{babel}
\usepackage{csquotes}
\usepackage[style=apa]{biblatex}
\usepackage[margin=1in]{geometry}
\usepackage[]{setspace}
% BEGIN: 8f7d3c5j4d9e
% \usepackage[]{hyperref}
\usepackage{xcolor}

% Define a new color for the link highlighting
\definecolor{linkhighlight}{RGB}{255, 255, 0}
\usepackage{fancyhdr} 

\pagestyle{fancy}
\fancyhf{}
\fancyheadoffset{0cm}
\renewcommand{\headrulewidth}{0pt} 
\renewcommand{\footrulewidth}{0pt}
\fancyhead[R]{\thepage}
\fancypagestyle{plain}{%
  \fancyhf{}%
  \fancyhead[R]{\thepage}%
}
% Set the link highlighting color to yellow
% \hypersetup{
%     colorlinks=true,
%     linkcolor=blue,
%     filecolor=magenta,      
%     urlcolor=black,
%     linkbordercolor=linkhighlight,
%     citebordercolor=linkhighlight,
%     urlbordercolor=linkhighlight,
%     pdfborderstyle={/S/U/W 1}
% }
\newrefcontext{sorting=name}
% END: 8f7d3c5j4d9e
\DeclareLanguageMapping{english}{english-apa}
\addbibresource{annot.bib} 

\title{\fontsize{18}{20}\selectfont\textbf{Assistive Technology and Affordability}}
\date{September 2023 \\\phantom{.}\\ Word Count: 1,110}

\usepackage[skip=0pt plus1pt, indent=72pt]{parskip}


\author{AP Seminar}


\begin{document}
\maketitle\newpage

\doublespacing\begin{raggedright}\parindent=36ptAccording to The American Psychological Association, 
    Independence is the state of having freedom from the influence 
    or control of other individuals or groups \parencite*[]{APA} and as I.S. Kon 
    States, there is no other personal quality as valuable as a 
    person’s independence \parencite{doi:10.2753/RES1060-9393310957}. A person’s independence relies on their ability to make and carry out decisions 
    independently, without assistance from another individual \parencite{doi:10.2753/RES1060-9393310957}. For many, having 
    independence is simple in the early and early middle stages of life.
    But as people age into the late stages of life, Their independence may 
    need to be assisted by external devices, or more formally, assistive 
    technology. While many devices help individuals 
    maintain their independence, There are barriers to their access, 
    with the largest being cost. Technologies that improve the 
    independence of individuals who are differently abled can be made 
    affordable through modern technologies and efficient manufacturing.

    \parindent=36pt
    The need for affordable assistive technology is greatly amplified in resource-limited
    locations like Bangladesh. In a scholarly and peer reviewed work, Borg and {\"O}stergren's study brings light to this issue by revealing that
    despite the awareness of the availability of assistive technology, affordability remains the main obstacle
    to their widespread adoption in the area \parencite{doi:10.3109/17483107.2014.974221}. In other nations,
    social protection systems are implemented to provide assistive technology to those who need it. These protections
    may be in the form of insurance, universal health coverage, or direct provision. However, these programs are often
    insufficient and leave many users with a financial burden, putting them at risk of poverty \parencite{doi:10.1080/10400435.2021.1994052}.
    It is essential to develop assistive technology that can be better afforded by social protection programs or by the users themselves to
    prevent them from falling into poverty and to allow them to maintain autonomy.

    \parindent=36pt
    One of the first steps in making assistive technologies more affordable lies 
    in the design of the product. In recent times, minimalist design has been experiencing a resurgence, calling back to its mid-19th century origins.  It can be seen everywhere,
    from Apple's website to the design of the new Tesla Cybertruck \parencite*{arts909874}. This idea can be applied to
    assistive technology as well. A simpler design results in fewer materials and less time in 
    the manufacture of the device. The combination of these two factors can lead
    to a lower cost of production and, therefore, a lower cost of the device. The use of simpler 
    manufacturing in assistive technology can be best seen in an engineering design report by the 
    TTK Center for Rehabilitation Research and Device Development (R2D2), and published in the journal \textit{Disability and Rehabilitation: Assistive Technology}. In their report, the team 
    detailed that in the team's third iteration of a manual standing wheelchair, the minimalistic design of the 
    wheelchair allowed for lower manufacturing and tooling costs, as well as making the user more 
    visible to others while using the device \parencite{doi:10.1080/17483107.2021.1892839}. In 
    the production-ready iteration of the wheelchair, the team was able to reduce its cost to around 
    210 United States Dollars. When released to the public, this reduction in
    cost can make the device more accessible to a wider range of individuals who require assistive technology.

    \parindent=36pt
    The widespread adaptation of low-cost microcontrollers and single-board computers has 
    heavily impacted the field of technology. These devices are increasingly becoming more affordable and can be used for a
    wide range of applications. Some applications include the collection of data for monitoring 
    and automated crop management. The versatility, in combination with the low cost of these devices, 
    opens up a wide range of possibilities for their use in assistive technology. In a case study
    presented by the Lassonde School of Engineering and published by Springer International, a team of students worked to develop an 
    assistive device that would allow those who are challenged by verbal communication to communicate their needs better. Their goal for the device was to make it an affordable alternative 
    to commercialized devices by prioritizing the use of durable, customizable, low-cost, and open-source 
    components. In the final iteration of the assistive device, the team was able to achieve all of their 
    goals using an openly designed accessibility-enhanced human interface device, a Raspberry Pi, and a speaker. 
    The total price of the device was around 80 Canadian dollars, with all the assembly instructions 
    and code being readily available online \parencite{10.1007/978-3-319-08599-9_44}. One benefit of using open-source components is that they can be manufactured by anyone and modified to fit the needs of the user.

    \parindent=36pt
    New Developments in assistive technology can take advantage of microcontrollers to create
    more advanced products while maintaining a low cost. An example of a device that takes advantage of this is the
    development of a robotic smart cane for the visually impaired. This product incorporates an Arduino microcontroller, an array of components, and a DC motor to navigate the user around obstacles and enhance its capabilities
    through a smartphone application. The smartphone application allows someone, such as a family member or 
    caregiver, to set a destination for the cane to navigate the user to. The combined cost of the components used in the 
    device is around 60–80 United States Dollars, making it an advanced yet affordable option for the visually impaired \parencite{10.1007/978-3-030-49282-3_36}.

    \parindent=36pt
    Another example of a device that takes advantage of single-board computing solutions is found in the design of smart 
    glasses for those who are visually impaired. A conference paper presented at the 2022 International Conference on Communication, Computing, and Internet of Things (IC3IoT), 
    describes the development process behind the device. The device is designed to assist people who are visually impaired in recognizing people around them through the use of a camera,
    a microphone, and a speaker connected to a Raspberry Pi and ESP32 microcontroller. The device features many algorithms that allow facial and recognition detection, hot word 
    detection, and text-to-speech. The device is also capable of connecting to the internet, allowing the user to access a wide range of information from Google and other services. The device's total cost
    can be approximated to 150 United States Dollars, making it another highly advanced yet cost-effective option for the visually impaired \parencite{9768012}.

    \parindent=36pt
    To conclude, the development of affordable assistive technology is essential to ensuring that individuals, especially those who are aging or in resource-limited locations, can maintain their independence. 
    By making use of minimalist design principles and the endless possibilities of low-cost microcontrollers in conjunction with open-source components, the realm of 
    the field of assistive technology can make significant strides toward providing a more inclusive and equitable future. This will allow for more individuals to maintain their independence and live a more fulfilling life.
\end{raggedright}



\pagebreak
\defbibheading{myheading}[References]{\centering\large\textbf{#1}}
\printbibliography[heading=myheading]
\end{document}
